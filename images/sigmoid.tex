\documentclass[varwidth=false, border=2pt]{standalone}

\usepackage{pgfplots}
\usepackage{tikz}

\begin{document}
\begin{tikzpicture}
    \begin{axis}[
        axis x line=bottom,
        axis y line=middle,
        x tick label style={/pgf/number format/fixed,
                            /pgf/number format/fixed zerofill,
                            /pgf/number format/precision=1},
        y tick label style={/pgf/number format/fixed,
                            /pgf/number format/fixed zerofill,
                            /pgf/number format/precision=1},
        ytick={0,0.5,1},
        xtick={-8,-4,0,4,8},
        xmin=-8,     % start the diagram at this x-coordinate
        xmax= 8,    % end   the diagram at this x-coordinate
        ymin= 0,     % start the diagram at this y-coordinate
        ymax= 1,   % end   the diagram at this y-coordinate
        %axis background/.style={fill=white},
        xlabel=$x$,
        ylabel=$s(x)$,
        tick align=outside,
        enlargelimits=false]
      % plot the stirling-formulae
      \addplot[domain=-8:8, red, ultra thick, samples=500] {1/(1+exp(-1*x))};
      %\addplot[domain=-4:4, blue, ultra thick] {1/(1+exp(-10*x))};
    \end{axis}
\end{tikzpicture}
\end{document}
